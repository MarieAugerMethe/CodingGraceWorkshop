Exploring US Healthcare data
April 13, 2014
By - r

(This article was first published on - r, and kindly contributed to R-bloggers)
A few days ago, the Centers for Medicare and Medicaid Services (CMS) released some unprecedented data on the US healthcare system. The data consists of 9 million rows showing how much each doctor in the US charged Medicare, for what, and how much Medicare paid out. It doesn't quite cover everything (for example, services with less than 11 beneficiaries were removed for privacy reasons), but its the best thing we've got.

Immediately after the release, we started seeing news stories about how some doctors were making millions of dollars. This information is easily found, and easily sensationalized, but I started to wonder what less obvious things might be in the data.

%=================================================================================== %
\subsection*{Getting the data}
You can grab the data here.

I decided to use R to analyze it, because of the ease of interactive exploration and making visualizations. You can get the code used throughout this post here.

Actually working with the data can be a bit tricky, depending on how much RAM you have. I have a good amount, and for convenience just read the whole csv file in with read.csv. You could always use something like the ff package or read the data straight into a database if you have memory limitations.

\begin{framed}
\begin{verbatim}
pm = read.delim(physician_Medicare, stringsAsFactors=FALSE)
\end{verbatim}
\end{framed}

Great, now that we have our data, let's explore it.
%==================================================================================== %
\subsection[Surface level explorations}
It's good to explore this kind of dataset rather than starting with specific questions to answer.

Here is a truncated view of the data:
\begin{framed}
\begin{verbatim}
         npi hcpcs_code line_srvc_cnt average_Medicare_payment_amt
2 1003000126 99222 115 108.11565
3 1003000126 99223 93 158.87000
4 1003000126 99231 111 30.72072
5 1003000126 99232 544 56.65566
6 1003000126 99233 75 81.39000
7 1003000126 99238 95 55.76884
\end{verbatim}
\end{framed}
The important columns are npi, which is a unique ID for the physician, and hcpcs\_code, which is a unique ID for the service the doctor performed. The other two fields will be important down the line. line\_srvc\_count is how many of the given service the doctor performed, and average\_Medicare\_payment\_amt is how much Medicare paid each time it was performed. You should look at the data documentation for what the other fields that aren't shown are.

Right now, the data is unique when npi and hcpcs\_code are taken together. So we have a summary of each service that each doctor performed.

We need to turn this into something that is unique on npi -- something that is a summary of what each doctor did. This will make comparison and finding useful things much simpler.

\subsection*{Converting the data}
To convert the data, we could use something like ddply or the base function by. The problem with these is that they will be very slow for 9 million rows. Even solutions like ff or bigmemory won't help much. We could read the data into a database and then do a group by query to get data in batches, but we already picked the lazy route of reading into memory.

Fortunately, the data table package for R is awesome, and will make what we are doing easy.

\begin{framed}
\begin{verbatim}
pm = data.table(pm)
phys_summ = pm[
  ,
  list(
    service_total=sum(line_srvc_cnt),
    ben_total=sum(bene_unique_cnt),
    payment=sum(average_Medicare_payment_amt * line_srvc_cnt),
    charged=sum(average_submitted_chrg_amt * line_srvc_cnt),
    allowed=sum(average_Medicare_allowed_amt * line_srvc_cnt),
    unique_services_per_patient=sum(bene_day_srvc_cnt)/sum(bene_unique_cnt),
    duplicates_per_service=sum(line_srvc_cnt)/sum(bene_day_srvc_cnt),
    services_per_patient=sum(line_srvc_cnt)/sum(bene_unique_cnt)
  ),
  by="npi"
  ]

\end{verbatim}
\end{framed}
The above code will transform our data so that npi is unique. It will calculate some descriptor variables (feel free to add your own) while it does it. In this case, we will see how much each doctor charged Medicare, how much they were paid, how many beneficiaries they served, and more.

Finally, some graphs!
Media reports focused on how much the top doctors made, and inequality in general is an interesting approach to this data. Let's look at income inequality among doctors by finding what percentage of the income is made by what percentage of doctors (ie top 5\% makes 50\% of the income).

\subsection*{Inequality across all physicians}
To do this, we need to first just extract the data for doctors, and then calculate the cumulative sum of doctor payments.

I was actually simplifying things before when I said that the data showed how much each doctor charged Medicare. The data actually has information from organizations (labs, hospitals, etc), as well as doctors. We can filter each one like this:

\begin{framed}
\begin{verbatim}
docs = phys_summ[nppes_entity_code=="I"]
orgs = phys_summ[nppes_entity_code=="O"]
nppes_entity_code indicates whether an individual or organization made the charges. We can now calculate our cumulative payments:

phys_ord = docs[order(docs$payment),c("npi", "payment"), with=FALSE]
phys_ord$pay_cumulative = cumsum(phys_ord$payment)
split_dist = floor(nrow(phys_ord)/20)
groups = as.numeric(lapply(1:20, function(x){
  phys_ord$pay_cumulative[split_dist * x]
}))
groups = (groups/groups[20]) * 100
\end{verbatim}
\end{framed}

The above code will give us how much the first 5\% of doctors made, how much the first 10\% made, and so on.

The above plot shows how stark the inequality is. The bottom 75\% of doctors get 25\% of the payments. The top 5\% get 47\% of all payments. Doctors in this case also includes nurses and other practitioners who get Medicare reimbursement, along with doctors who don't bill Medicare much, so these numbers are likely too extreme.

Gender based numeric inequality
We can also look at the data by gender. Overall, there are 523085 male physicians, and 302023 female.

Let's break down most common occupations by gender.

\begin{framed}
\begin{verbatim}

common_occupations = names(tail(sort(table(docs$provider_type)),15))
occupations = do.call(rbind,lapply(common_occupations, function(x){
  data.frame(occupation=x, female_count=nrow(females[females$provider_type==x,]), male_count=nrow(males[males$provider_type==x,]))
}))
\end{verbatim}
\end{framed}
This will give us a list of the 10 most common occupations, and how many males and females are doing them. Let's make a chart:

We see that women outnumber men in jobs like Nurse Practitioner and Physician Assistant, whereas there are more men in jobs like Internal Medicine and Emergency Medicine.

\subsection*{Gender based payment inequality}
This is interesting, but doesn't tell us much about payments. Let's look at how much practitioners of each speciality are paid, broken down by gender:

Men are, on average, reimbursed more from Medicare for every single speciality in the top 10 most common. This is kind of insane, and I don't know how to explain it. Anyone with insight here would be welcome.

So where are all these doctors, anyways?
Let's move from high-level analysis into location based analysis. One interesting way to do this is to see where the "million dollar doctors" -- the ones who bill the most to Medicare -- are. Let's make a map.

The lesson seems to be, if you are a doctor, get to Florida ASAP.

\subsection*{State life expectancy}
Let's look at life expectancy by state and see how that correlates with Medicare spending. Two theories would be that spending is higher in states with a lower life expectancy (because it is more needed there). The opposite could also be true (spending is higher in states with higher life expectancy, which is what leads to the higher expectancy).

In order to do this, we can get life expectancy data from here. It is pretty easy to copy/paste the data into a csv file, or use an automated scraper.

We can then create a per state charges data frame:

\begin{framed}
\begin{verbatim}
state_data = tapply(docs$payment, docs$nppes_provider_state, mean)
per_state_charges = data.frame(state=names(state_data), charge=state_data)
And we can read in the life expectancy data and combine them (assuming we read the data into life_e):

life_comp = merge(life_e, per_state_charges, by.x="Code", by.y="state", all.y=FALSE, all.x=TRUE)
We can then look at the correlation between life expectancy and average Medicare payments by state.

groups = c("African.American", "Asian", "Latino", "Native.American", "White", "Total")
correlations = as.numeric(lapply(groups, functioExploring US Healthcare data

\end{verbatim}

%============================================================================================================================ %
\subsection*{State life expectancy}
Let's look at life expectancy by state and see how that correlates with Medicare spending. Two theories would be that spending is higher in states with a lower life expectancy (because it is more needed there). The opposite could also be true (spending is higher in states with higher life expectancy, which is what leads to the higher expectancy).

In order to do this, we can get life expectancy data from here. It is pretty easy to copy/paste the data into a csv file, or use an automated scraper.

We can then create a per state charges data frame:
\begin{framed}
\begin{verbatim}
state_data = tapply(docs$payment, docs$nppes_provider_state, mean)
per_state_charges = data.frame(state=names(state_data), charge=state_data)
And we can read in the life expectancy data and combine them (assuming we read the data into life_e):

life_comp = merge(life_e, per_state_charges, by.x="Code", by.y="state", all.y=FALSE, all.x=TRUE)
We can then look at the correlation between life expectancy and average Medicare payments by state.

groups = c("African.American", "Asian", "Latino", "Native.American", "White", "Total")
correlations = as.numeric(lapply(groups, function(x){
	dat = life_comp[!is.na(life_comp[,x]),]
	cor(dat[,x], dat$charge)
}))
cor_frame=data.frame(group=gsub("\.", " ", groups), cor=correlations)
\end{verbatim}
\end{framed}
%================================================================= %
This will actually find the correlation for each racial group (the life expectancy data has it, so why not use it?).

This is a really interesting result! The total is negatively correlated with life expectancy very strongly, which indicates that Medicare spending is higher where it is needed (ie states with lower life expectancies get more Medicare spending).

As for the individual ethnic groups, I am not 100\% certain what it means, but I will try to interpret (let me know what you think!). The interesting part is that life expectancy for whites correlates more strongly with spending, which indicates that Medicare is more strongly optimized towards the needs of the white population than the population as a whole. Other groups are less strongly negatively correlated, and some are positively correlated, which indicates that Medicare isn't as responsive to those groups. 

Of course, the data is only 100\% complete for whites and overall, so missing data may be causing noise. But it is very interesting that Medicare spends much less in areas where the native american life expectancy is lowest, for instance.

\subsection*{The End / Future analysis}
I really enjoyed this analysis, and want to do more, but I am running out of weekend time in which to do it! Next time, I think I will take a look at fraud, and see if it is possible to make models to detect fraud. I also want to see how combining this dataset with some of the other interesting Medicare datasets will look.n(x){
  dat = life_comp[!is.na(life_comp[,x]),]
  cor(dat[,x], dat$charge)
}))
cor_frame=data.frame(group=gsub("\.", " ", groups), cor=correlations)
This will actually find the correlation for each racial group (the life expectancy data has it, so why not use it?).

This is a really interesting result! The total is negatively correlated with life expectancy very strongly, which indicates that Medicare spending is higher where it is needed (ie states with lower life expectancies get more Medicare spending).

As for the individual ethnic groups, I am not 100% certain what it means, but I will try to interpret (let me know what you think!). The interesting part is that life expectancy for whites correlates more strongly with spending, which indicates that Medicare is more strongly optimized towards the needs of the white population than the population as a whole. Other groups are less strongly negatively correlated, and some are positively correlated, which indicates that Medicare isn't as responsive to those groups. Of course, the data is only 100% complete for whites and overall, so missing data may be causing noise. But it is very interesting that Medicare spends much less in areas where the native american life expectancy is lowest, for instance.

The End / Future analysis
I really enjoyed this analysis, and want to do more, but I am running out of weekend time in which to do it! Next time, I think I will take a look at fraud, and see if it is possible to make models to detect fraud. I also want to see how combining this dataset with some of the other interesting Medicare datasets will look.