
\documentclass{article}

\usepackage{amssymb}
\usepackage{amsmath}
\usepackage{graphicx}
\usepackage{framed}

\hoffset=0pt
\oddsidemargin=15pt
\voffset=0pt
\marginparwidth = 2pt
\textheight = 642pt
\textwidth = 420pt


\begin{document}
\section{Multiple table verbs}

As well as verbs that work on a single \texttt{tbl}, there are also a set of useful verbs that work with two \texttt{tbl}s at a time: joins and set operations.

\subsection*{Joins}
dplyr implements the four most useful joins from SQL:

\begin{itemize}
\item \texttt{inner\_join(x, y)}: matching x + y
\item \texttt{left\_join(x, y)}: all x + matching y
\item \texttt{semi\_join(x, y)}: all x with match in y
\item \texttt{anti\_join(x, y)}: all x without match in y
\end{itemize}

\subsection*{Set Theory Operations}

dplyr implements the methods for set theory operations

\begin{itemize}
\item \texttt{intersect(x, y)}: all rows in both x and y
\item \texttt{union(x, y)}: rows in either x or y
\item \texttt{setdiff(x, y)}: rows in x, but not y
\end{itemize}
%================================================================= %
\end{document}