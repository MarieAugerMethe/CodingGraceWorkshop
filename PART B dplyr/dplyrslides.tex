\documentclass{beamer}

\usepackage{graphicx}
\usepackage{amsmath}
\usepackage{framed}

\begin{document}
	
	% = http://cran.rstudio.com/web/packages/dplyr/vignettes/introduction.html
	
\begin{frame}[fragile]
\frametitle{dplyr : Grammar of data manipulation}
\begin{itemize}
	\item dplyr  is mainly authored by Hadley Wickham and Romain Francois. It is designed to be intuitive and easy to learn, thereby making “doing things” in \texttt{R} more user friendly.
	\item dplyr is a new package which provides a set of tools for efficiently manipulating datasets in \texttt{R}.
	\item dplyr is the next iteration of plyr, focussing on only data frames. \item dplyr is faster, has a more consistent API and should be easier to use. 
\end{itemize}
\end{frame}
%=================================================================== %

\begin{frame}
\frametitle{dplyr : abstract by Hadley Wickham}	
	
	\noindent There are three key ideas that underlie dplyr:
	
	\begin{enumerate}
		\item[1] Your time is important, so Romain Francois has written the key pieces in Rcpp to provide blazing fast performance. Performance will only get better over time, especially once we figure out the best way to make the most of multiple processors. 
\end{enumerate}
\end{frame}
%========================================================== %
\begin{frame}
\frametitle{dplyr : abstract by Hadley Wickham}	

\begin{enumerate}
		\item[2] Tabular data is tabular data regardless of where it lives, so you should use the same functions to work with it. 
		
		With dplyr, anything you can do to a local data frame you can also do to a remote database table. PostgreSQL, MySQL, SQLite and Google bigquery support is built-in; adding a new backend is a matter of implementing a handful of S3 methods. 
\end{enumerate}
\end{frame}
%========================================================== %
\begin{frame}
\frametitle{dplyr : abstract by Hadley Wickham}	

	\begin{enumerate}
		\item[3] The bottleneck in most data analyses is the time it takes for you to figure out what to do with your data, and dplyr makes this easier by having individual functions that correspond to the most common operations  group\_by, summarise,mutate, filter, select and arrange). Each function does one only thing, but does it well.
	\end{enumerate}
\end{frame}
%======================================================================================= %

\begin{frame}
	
	\frametitle{Working with dplyr} \textbf{dplyr} focussed on tools for working with data frames (hence the d in the name). 
\textbf{dplyr} has three main goals:

\begin{itemize}
	\item Identify the most important data manipulation tools needed for data analysis and make them easy to use from \texttt{R}.
	
	\item Provide very fast performance for in-memory data by writing key pieces in C++.
	
	\item Use the same interface to work with data no matter where it's stored, whether in a data frame, a data table or database.
\end{itemize}
\end{frame}
%\subsection{Installing dplyr}
%You can install the latest released version from CRAN with the code below.
%You can also install and load the data packages used in most examples: 
%\begin{framed}
%	\begin{verbatim}
%	install.packages("dplyr")
%	install.packages(c("nycflights13", "Lahman"))
%	
%	library(dplyr) # for functions
%	library(nycflights13) # for data
%	\end{verbatim}
%\end{framed}
%===================================================================== %
\begin{frame}
\frametitle{Tidy Data}
To make the most of dplyr, Hadley Wickham recommends that you familiarise yourself with the \textbf{principles of tidy data}. This will help you get your data into a form that works well with \textbf{dplyr}, \textbf{ggplot2} and \texttt{R}'s many modelling functions.\\
\end{frame}

%====================================================================== %
\begin{frame}[fragile]
\begin{framed}
	\noindent Three Principles from Hadley Wickham's paper
	\begin{itemize}
		\item[1.] Each variable forms a column, 
		\item[2.] Each observation forms a row, 
		\item[3.] Each table/file stores data about one kind of observation.
	\end{itemize}
\end{framed}
\noindent \textbf{Remark:}  The paper ``\textit{\textbf{Tidy data}}" by Hadley Wickham (RStudio) can be downloaded from 
\begin{verbatim}
http://vita.had.co.nz/papers/tidy-data.pdf
\end{verbatim}
\end{frame}
%=================================================================== %
\begin{frame}
\frametitle{Key data structures}

The key object in \textbf{dplyr} is a \texttt{tbl}, a representation of a tabular data structure. Currently dplyr supports:

\begin{itemize}
	\item data frames - the  most commonly encountered R data structure. 
	\item data tables - a data structure that is designed for intensive data analysis.
\end{itemize}

%\noindent For this class, We will concentrate on \textbf{dplyr} exercises with data frames mostly. However we would advise you to try and learn more about working with data tables in the future.\\
%\bigskip
\end{frame}
%============================================================================= %
\begin{frame}
\noindent For advanced users, \textbf{dplyr} also supports the following databases: \textit{SQLite, PostgreSQL, Redshift, MySQL/MariaDB, Bigquery, MonetDB} and data cubes with arrays (partial implementation). We will not cover those topics in this workshop.
\end{frame}
%============================================================================= %
\begin{frame}
	\frametitle{CRAN tutorial}
\begin{figure}
\centering
\includegraphics[width=1.05\linewidth]{"C:/Users/Kevin/Documents/GitHub/CodingGraceWorkshop/PART B dplyr/introdplyr"}

\end{figure}

\end{frame}
%=======================================================================================%
\begin{frame}[fragile]
\frametitle{Installing dplyr}
Straightforward R package installation.
\begin{framed}
\begin{verbatim}
install.packages("dplyr")
library(dplyr)

# Data Set Examples
# 1. iris
# 2. mtcars

\end{verbatim}
\end{framed}
\end{frame}
%===================================================================== %
	
	\begin{frame}[fragile]
		\frametitle{iris data set}
\begin{framed}
		\begin{verbatim}
		> names(iris)
		[1] "Sepal.Length"
		[2] "Sepal.Width" 
		[3] "Petal.Length"
		[4] "Petal.Width" 
		[5] "Species"    
		\end{verbatim}
\end{framed}
	\end{frame}
%============================================================================ %

\begin{frame}[fragile]
\frametitle{mtcars data set}
\begin{verbatim}
> names(mtcars)
[1] "mpg"  "cyl"  "disp" "hp"  
[5] "drat" "wt"   "qsec" "vs"  
[9] "am"   "gear" "carb"
\end{verbatim}
\end{frame}
	%===================================================================================%
	\begin{frame}[fragile]
	\frametitle{Example Data Sets}
		\begin{verbatim}
		dim(iris)
		class(iris)
		mode(iris)
		
			dim(mtcars)
			class(mtcars)
			mode(mtcars)
		\end{verbatim}
	\end{frame}
%==================================================================================== %
\begin{frame}
\frametitle{dplyr; Single Table Verbs}
	\begin{figure}
\centering
\includegraphics[width=0.7\linewidth]{singletableverbs}

\end{figure}

\end{frame}
%=========================================================================================%
\begin{frame}[fragile]
\frametitle{Grouping with the \texttt{group\_by} command}
\begin{verbatim}
	
iris.sp <- group_by(iris,Species)
class(iris.sp)
		
summarise(iris.sp,mean(Sepal.Length), sd(Petal.Length))

\end{verbatim}
\end{frame}
%==================================================================== %
\begin{frame}
\begin{figure}
\centering
\includegraphics[width=1.05\linewidth]{irisgroupby}
\end{figure}
\end{frame}	

%============================================================================== %
\begin{frame}
\begin{figure}
\centering
\includegraphics[width=1.1\linewidth]{mtcarssummarise}
\label{fig:mtcarssummarise}
\end{figure}
\end{frame}
		
	
	%=========================================================================================%
\begin{frame}
\frametitle{Filter rows with \texttt{filter()}}
\begin{itemize}
\item \texttt{filter()} allows you to select a subset of the rows of a data frame. 
\item The first argument is the name of the data frame, and the second and subsequent are filtering expressions evaluated in the context of that data frame.
\end{itemize}



\end{frame}
%=========================================================================================%
\begin{frame}[fragile]	
\begin{framed}
\begin{verbatim}
	iris.vir1 <- filter(iris,Species=="virginica")
	
	# Species is Virginica OR Petal.length is greater than 3.2
	
	iris.vir2 <- filter(iris,Species=="virginica" | Petal.Length >3.2)
	
	iris.vir3 <- filter(iris,Species=="virginica" & Petal.Length >3.9)
	
\end{verbatim}
\end{framed}
\end{frame}
%========================================================================================%
\begin{frame}
\frametitle{Ordering Data Sets with \texttt{arrange()}}
\begin{itemize}
\item \texttt{arrange()} works similarly to \texttt{filter()} except that instead of filtering or selecting rows, it reorders them. 

\item It takes a data frame, and a set of column names (or more complicated expressions) to order by.

\item If you provide more than one column name, each additional column will be used to break ties in the values of preceding columns.

\item Use \texttt{desc()} (or \texttt{rev()})to order a column in descending order:

%arrange(flights, desc(arr_delay))
\end{itemize}
\end{frame}
%===================================================================================== %
\begin{frame}
	\begin{figure}
		\centering
		\includegraphics[width=0.97\linewidth]{irisarrange}
		
	\end{figure}
	
\end{frame}

%==================================================================================== %

\begin{frame}
	\begin{figure}
		\centering
		\includegraphics[width=0.97\linewidth]{irisarrange2}
		
	\end{figure}
	
\end{frame}

% Graphics: irisarrange

%=====================================================================================%
\begin{frame}
\frametitle{Select columns with \texttt{select()}}
\begin{itemize}
\item Often you work with large datasets with many columns where only a few are actually of interest to you. 
\item select() allows you to rapidly zoom in on a useful subset using operations that usually only work on numeric variable positions.
\end{itemize}
\end{frame}

%=======================================================================================%
\begin{frame}[fragile]
\frametitle{Add new columns with \texttt{mutate()} }

As well as selecting from the set of existing columns, it’s often useful to add new columns that are functions of existing columns. This is the job of \texttt{mutate()}:

\begin{framed}
\begin{verbatim}
iris2 =  mutate(iris, PW2 = log(Petal.Width), PL2=sqrt(Petal.Length) )
head(iris2)
\end{verbatim}
\end{framed}
\end{frame}
%================================================================================= %
%% Graph irismutate
\begin{frame}
	
	\frametitle{Add new columns with \texttt{mutate()} }
	\begin{figure}
\centering
\includegraphics[width=0.9\linewidth]{irismutate}

\end{figure}

\end{frame}
%==========================================================================%
\begin{frame}[fragile]
	
	\frametitle{Add new columns with \texttt{mutate()} }
\texttt{mutate} allows you to refer to columns that you just created:

\begin{verbatim}
iris3 =  mutate(iris, 
     PW2 = log(Petal.Width), 
     PL2=sqrt(Petal.Length), 
     Ratio=PL2/PW2 )
     
head(iris3)
\end{verbatim}
\end{frame} 
%================================================================== %
\begin{frame}

\frametitle{Add new columns with \texttt{mutate()} }
	\begin{figure}
		\centering
		\includegraphics[width=0.9\linewidth]{irismutate2}
		
	\end{figure}
	
\end{frame}


%% Graphic irismutate2




\end{document}
